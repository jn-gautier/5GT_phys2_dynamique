\section{Exercices supplémentaires}
\begin{exercise}
    Un coureur professionnel peut accélérer sur le starting-block avec une accélération quasi horizontale de \(a=15\unit{[m\cdot s^{-2}]}\).
    \begin{enumerate}[a)]
        \item Quelle est la force qu'un sprinteur de \(45\unit{[kg]}\)  doit exercer sur le starting-block pour produire cette accélération ?
        \item Quel corps exerce la force qui propulse le coureur ? Le coureur lui-même ou le starting-block ?
    \end{enumerate}

\end{exercise}

\begin{exercise}
    Le plancher d'un ascenseur exerce une force verticale vers le haut de \(650 \unit{N}\) sur un passager dont le poids vaut \(650 \unit{N}\).
    \begin{enumerate}[a)]
        \item Le passager est-il au repos, en MRU ou en MRUA ?
        \item S'il est en MRUA, dans quel sens accélère-t-il ?
    \end{enumerate}
\end{exercise}

\begin{exercise}
    Le plancher d'un ascenseur exerce une force verticale vers le haut de \(620 \unit{N}\) sur un passager dont le poids vaut \(650 \unit{N}\).
    \begin{enumerate}[a)]
        \item Le passager est-il au repos, en MRU ou en MRUA ?
        \item S'il est en MRUA, dans quel sens accélère-t-il ?
    \end{enumerate}
\end{exercise}

\begin{exercise}
    Un électron démarre avec une vitesse nulle à l'extrémité d'un tube cathodique (un dispositif utilisé pour accélérer les électrons dans les anciennes télévisions). Il se déplace en ligne droite dans le dispositif d'accélération long de 1,8[cm] et arrive finalement sur l'écran de la télévision avec une vitesse de \(3 \times 10^{6}[m \cdot s^{-1}]\).
    \begin{enumerate}[a)]
        \item Quelle est la valeur de l'accélération de l'électron ?
        \item Quelle est la durée de cette accélération ?
        \item Quelle est la valeur de la force qui s'exerce sur l'électron ?
    \end{enumerate}
    \(m_{e^-}=9,11 \times 10^{-31}[kg]\)
\end{exercise}
\begin{solution}
    \begin{itemize}
        \item \(a=2,5 \times 10^{14} [m \cdot s^{-2}]\)
        \item \(\Delta t=1,2 \times 10^{-8}[s]\)
        \item \(F=2,278 \times 10^{-16} [N]\)
    \end{itemize}
\end{solution}

\begin{exercise}
    Un patineur de \(68,5[kg]\) se déplace à la vitesse de \(2,4[m \cdot s^{-1}]\) sur de la glace. Lorsqu'il se laisse glisser sur la glace, il prend \(3,52[s]\) pour s'arrêter à cause des forces de frottement. Quelle est la valeur des forces de frottement ?
\end{exercise}

\begin{exercise}
    Une femme se tient debout sur un pèse-personne placé dans un ascenseur accélérant vers le haut avec une intensité de \(0,8[m \cdot s^{-2}]\). Si le poids de cette passagère est de \(750[N]\), quelle sera la valeur indiquée sur le pèse-personne ? Réalise un schéma sur lequel tu place la femme et le pèse-personne ainsi que l'ensemble des forces s'exerçant sur ces deux corps.
\end{exercise}

\begin{exercise}
    Si, suite à un accident l'ascenseur de la question précédente tombe avec une accélération de \(5[m \cdot s^{-2}]\). Quelle sera la valeur indiquée par le pèse-personne ?
\end{exercise}

\begin{exercise}
    Que peux-tu prévoir si l'ascenseur tombe en chute libre, c'est-à-dire avec une accélération de \(9,81[m \cdot s^{-2}]\) ? Quelle valeur sera indiquée par le pèse-personne ? Comment l'occupante va-t-elle se comporter\footnote{Aucun passager d'ascenseur n'a été blessé durant la réalisation de cet exercice... Aucun ascenseur non plus.} ?
\end{exercise}